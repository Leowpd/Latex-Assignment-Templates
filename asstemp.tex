\documentclass[11pt, a4paper]{article}
\usepackage{alicetex}
\usepackage{parskip}
\lstset{style=mainstyle}

\title{General Assignment Template}
\author{Leo Duncan \\ ldun202, 657579290}
\date{xxth Month 202x}

\begin{document}

\maketitle

\section{General Math}
\quest{Basic math examples for convenient copy-pasting}

\subsection{(a)} %Basic multiline equation
\quest{Basic multiline equation example.}

\begin{equation*}
    \begin{split}
        a &= b + c \\
        &= (b + c) \\
    \end{split}
\end{equation*}

\subsection{(b)} %Vectors and matrices
\quest{Vectors and matrices.}

Column vector square brackets: $\bm{x} = \colvec{1\\2\\3}$, round brackets: $\vec{v} = \colvecbr{4\\5\\6\\7}$.

Row vector square brackets: $\bm{u} = \colvec{1&2&3}$;

Round brackets: $\vec{w} = \colvecbr{4&5&6&7}$.

Matrix $A = \left[\begin{array}{cccc}
    a & b & c & d \\
    e & f & g & h \\
    i & j & k & l \\
\end{array}\right]$.

Another example $B = \left[\begin{array}{ccc|c}
    a & b & c & d \\
    e & f & g & h \\
    i & j & k & l \\
\end{array}\right]$.

\subsection{(c)} %Sums, limits, integration
\quest{Sums, limits, integration}

$$\sum \frac{1}{k+\sqrt{k}}$$

$$\sum\limits_{k=2}^\infty \frac{3^k-1}{4^k}$$

$$\lim_{k \to \infty} \frac{k^2 + k + 1}{2k^2}$$

$$\int_{-1}^{1} e^{x} \, dx$$

\subsection{(d)} %Using /frac and /dfrac
\quest{Using \textbackslash \texttt{frac} and \textbackslash \texttt{dfrac}.}

Using \textbackslash \texttt{frac} $\frac{k^2+1}{k^3+1}$ and using \textbackslash \texttt{dfrac} $\dfrac{k^2+1}{k^3+1}$ and using \textbackslash \texttt{tfrac} $\tfrac{k^2+1}{k^3+1}$.

Note that when in "big" equation environment, \$\$, they give the same result. Avoid using \textbackslash \texttt{dfrac} unless necessary.
\newpage


\section{Graphics}
\quest{Pictures and everything so exciting}

Here is an example of an image with a caption.

\begin{figure}[!h]
    \centering
    \includegraphics[width=0.5\textwidth]{images/clownfishplot1.png}
    \label{fig:f1}
    \caption{Clownfish model starting at population 31 (thousand).}
\end{figure}


Here is an example of two images side-by-side, with captions.

\begin{figure}[!h]
    \begin{subfigure}[b]{0.45\textwidth}
      \includegraphics[width=\textwidth]{images/clownfishplot2.png}
      \label{fig:f2}
      \caption{Cobweb plot $C_0 = 1$}
    \end{subfigure}
    \hfill
    \begin{subfigure}[b]{0.45\textwidth}
      \includegraphics[width=\textwidth]{images/clownfishplot3.png}
      \label{fig:f3}
      \caption{Cobweb plot $C_0 = 3$}
    \end{subfigure}
    \caption{Updated model cobweb plots.}
\end{figure}




\section{Tables}
\quest{Yep, tables, as the title would suggest}

\begin{table}[!ht]
    \centering
    \small
    \begin{tabular}{|c|c|}
        \hline 
        $Z_1$ & $Z_0$ \\ \hline
        0 & 0 \\ \hline
        0 & 1 \\ \hline
        1 & 0 \\ \hline
        1 & 1 \\ \hline
    \end{tabular}
    \label{Step (2) Table}
\end{table}


\begin{table}[!ht]
    \small
    \begin{tabular}{|c|c|c|c|c|}
        \hline 
         & QA/L1 & QB/L2 & QC/L3 & QD/L4 \\ \hline
        Starting State & 0 & 1 & 0 & 1 \\ \hline
        After 1 clock pulse & 1 & 0 & 1 & 0 \\ \hline
        After 2 clock pulses & 1 & 1 & 0 & 1 \\ \hline
        After 3 clock pulses & 1 & 1 & 1 & 0 \\ \hline
    \end{tabular}
    \label{Step (14) Table}
\end{table}




\section{Code}
\quest{Code code code code code}
%be sure to include pseudocode examples too (220 ass 1)

\lstset{style=mainstyle}
\begin{lstlisting}[language=MATLAB]
Seq = randi(2,1,1000); % generate random sequence of coin flips
k = 0;

% while-loop that runs until the desired sequence, THH, is found
while(Seq(k+1)~=2 || Seq(k+2)~=1 || Seq(k+3)~=1)
    k = k + 1;
end

disp(k) % output how many flips before the sequence THH
\end{lstlisting}


\section{Academic Writing} %ACTUALLY MAYBE PUT THESE FINAL TWO IN A SEPEATE FILE (MAYBE COMBINE WHEN DONE)

\quest{How to format big chunks of writing}




\section{Bibliography}
\quest{Referencing, footnotes, etc}





\end{document}